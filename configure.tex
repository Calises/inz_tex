%Szablon do pracy inżynierskiej 2014
% Autor: Mateusz Kraiński
% Kontakt: mt.krainski@gmail.com


\documentclass[a4paper,twoside,10pt]{report} 

\usepackage[a4paper, left=3.5cm, right=2.5cm, top=2.5cm, bottom=2.5cm, headsep=1.2cm]{geometry}
\setlength\parindent{1.25cm}

\usepackage[MeX]{polski} %Mex zawiera polskie fonty
\usepackage[utf8]{inputenc} %kodowanie 1250, dla iso to: "latin2"

% Closest to arial as i could get
\renewcommand{\familydefault}{\sfdefault}
\normalfont

\usepackage{textcomp}
\usepackage{pifont}
\usepackage{float}
\usepackage[pdftex]{graphicx}
\usepackage[nottoc]{tocbibind}
\usepackage{longtable}
\usepackage{enumerate}
\usepackage{fancyhdr}
\usepackage{indentfirst} %do każdego akapitu dodawane jest wcięcie, zgodnie z polskimi standardami drukarskimi.
\usepackage{sectsty}
\usepackage{setspace} %ustawianie interlini, dobrze wygladaja tabele, \singlespacing, \onehalfspacing, and \doublespacing
\usepackage{float}
\usepackage{amsthm} %dla twierdzen
\usepackage{amsmath} %pakiet matematyczny
\usepackage{amssymb} %pakiet dodatkowych symboli
\usepackage{multirow}\usepackage{amsthm} %dla twierdzen
\usepackage{amsmath} %pakiet matematyczny
\usepackage{amssymb} %pakiet dodatkowych symboli
\usepackage{multirow}
\usepackage{xcolor,listings} %listingi
\usepackage[final]{pdfpages}
\usepackage{url}

\usepackage{tabularx}

\usepackage{listings}
\usepackage{xcolor}
\lstset { %
    language=C++,
    backgroundcolor=\color{black!5}, % set backgroundcolor
    basicstyle=\footnotesize,% basic font setting
}

\singlespacing
\sectionfont{\large}
\subsectionfont{\normalsize}
\renewcommand{\baselinestretch}{1.5}



%---------------------------------------------------------------------------------
%Sortowanie tekstu
\usepackage{datatool}% http://ctan.org/pkg/datatool

\DTLnewdb{MathSymbols}
\DTLnewdb{Shortcuts}
\newcolumntype{R}{>{\raggedright\arraybackslash}X}

\newcommand{\mathitem}[2]{%
  \DTLnewrow{MathSymbols}% Create a new entry
  \DTLnewdbentry{MathSymbols}{name}{#1}% Add entry as description
  \DTLnewdbentry{MathSymbols}{description}{#2}
}
\newcommand{\shortitem}[2]{%
  \DTLnewrow{Shortcuts}% Create a new entry
  \DTLnewdbentry{Shortcuts}{name}{#1}% Add entry as description
  \DTLnewdbentry{Shortcuts}{description}{#2}
}
\newenvironment{sortedlist}{%
  \DTLifdbexists{MathSymbols}{\DTLcleardb{MathSymbols}}{\DTLnewdb{MathSymbols}}
  \DTLifdbexists{Shortcuts}{\DTLcleardb{Shortcuts}}{\DTLnewdb{Shortcuts}}
  % Create new/discard old list
  }
  {%
  \DTLsort{name}{MathSymbols}% Sort list
  \DTLsort{name}{Shortcuts}
  \begin{flushleft}
  \begin{tabularx}{\linewidth}[c]{l c R}
  \DTLforeach*{MathSymbols}{\theName=name, \theDesc=description}{
     \theName & -- & \theDesc \vspace{6pt}\\}\linebreak
  \DTLforeach*{Shortcuts}{\theName=name, \theDesc=description}{
     \theName & -- & \theDesc \vspace{6pt}\\}
  \end{tabularx}
  \vspace{-26pt}
  \end{flushleft}
}

\DTLnewdb{eqlist}

\newcommand{\EQDitem}[2]{%
  \DTLnewrow{eqlist}% Create a new entry
  \DTLnewdbentry{eqlist}{name}{#1}% Add entry as description
  \DTLnewdbentry{eqlist}{description}{#2}
}
\newenvironment{equationDescriptorS}{%
  \DTLifdbexists{eqlist}{\DTLcleardb{eqlist}}{\DTLnewdb{eqlist}}
  % Create new/discard old list
  }
  {%
  \DTLsort{name}{eqlist}% Sort list
  \begin{flushleft}
  \vspace{-8pt}
  \begin{tabularx}{\linewidth}[c]{l c R}
  \DTLforeach*{eqlist}{\theName=name, \theDesc=description}{
     \theName &--& \theDesc\\}
  \end{tabularx}
  \vspace{-24pt}
  \end{flushleft}
}

\newenvironment{equationDescriptor}{%
  \DTLifdbexists{eqlist}{\DTLcleardb{eqlist}}{\DTLnewdb{eqlist}}
  % Create new/discard old list
  }
  {
  \begin{flushleft}
  \vspace{-8pt}
  \begin{tabularx}{\linewidth}[c]{l c R}
  \DTLforeach*{eqlist}{\theName=name, \theDesc=description}{
     \theName &--& \theDesc\\}
  \end{tabularx}
  \vspace{-24pt}
  \end{flushleft}
}



%-------------------------------------------------------------sub--------------------
%Formatowanie spisu treści
\usepackage{lipsum}
\usepackage{titlesec}
\usepackage{titletoc}

% Definicja nowego sposobu umieszczania tytułów sekcji
\def\empty{none}
\newcommand{\namedchapter}[2][\empty]{\chapter[#2 \ifx#1\empty\else(#1)\fi]{#2}}
\newcommand{\namedsection}[2][\empty]{\section[#2 \ifx#1\empty\else(#1)\fi]{#2}}
\newcommand{\namedsubsection}[2][\empty]{\subsection[#2 \ifx#1\empty\else(#1)\fi]{#2}}


\titlecontents{chapter}
[0.0cm]             % left margin
{\vspace{6 pt }}                  % above code
{%                  % numbered format
{\thecontentslabel. }%
}%
{}         % unnumbered format
{\enspace\titlerule*{.}\contentspage}         % filler-page-format, e.g dots

\titlecontents{section}
[1em]             % left margin
{\vspace{6 pt }}                   % above code
{%                  % numbered format
{\thecontentslabel. }%
}%
{}         % unnumbered format
{\enspace\titlerule*{.} \contentspage}         % filler-page-format, e.g dots

% indented subsection (in toc)
\titlecontents{subsection}
[2em]             % left margin
{\vspace{6 pt }}                   % above code
{%                  % numbered format
{\thecontentslabel. }%
}%
{}         % unnumbered format
{ \enspace\titlerule*{.}\contentspage}         % filler-page-format





%---------------------------------------------------------------------------------
% Formatowanie rozdziałów, podrozdziałów itd.
\usepackage{textcase}
\usepackage{titlesec}


\titleformat
{\chapter} % command
[hang] % shape
{\bfseries\large} % format
{\thechapter. } % label
{0pt} % sep
{\MakeTextUppercase} % before-code
[] % after-code
\titlespacing*{\chapter}{0pt}{12pt}{6pt}


\titleformat
{\section} % command
[hang] % shape
{\normalsize\bfseries\itshape} % format
{\thesection. } % label
{0em} % sep
{} % before-code
[] % after-code
\titlespacing*{\section}{0pt}{12pt}{6pt}

\titleformat
{\subsection} % command
[hang] % shape
{\normalsize\itshape} % format
{\thesubsection. } % label
{0em} % sep
{} % before-code
[] % after-code
\titlespacing*{\subsection}{0pt}{12pt}{6pt}



%---------------------------------------------------------------------------------
% Change itemize point style
\renewcommand{\labelitemi}{$\bullet$}
\renewcommand{\labelitemii}{$\circ$}
\renewcommand{\labelenumi}{\arabic{enumi}. }
\renewcommand{\labelenumii}{\labelenumi\arabic{enumii}. }
\renewcommand{\labelenumiii}{\labelenumii\arabic{enumiii}. }




%---------------------------------------------------------------------------------

%Change image style, add simple functions InsertImageSimple and InsertSubImagesSimple
%\usepackage{caption}
\usepackage[font=small, labelsep=period, figurename=Rys., aboveskip=6pt, belowskip=0pt, justification=centering]{caption}
\usepackage{subcaption}

\captionsetup*[subfigure]{ 
	font=small, 
	singlelinecheck=off, 
	justification=raggedright, 
	labelformat=simple, 
	labelsep=none,
	aboveskip=6pt, 
	belowskip=0pt
} 

\graphicspath{ {rys/} }

\newcommand{\InsertImageSimple}[4][\empty]
{
\begin{figure}[h!tb]
    \centering
    \ifx#1\empty\includegraphics[scale=1]{#2}\else\includegraphics[width = #1\linewidth]{#2}\fi
    \caption{#3}
	\label{fig:#4}
\end{figure}
}

\newcommand{\InsertSubImagesSimple}[8]
{
\begin{figure}[h!tb] 
	\centering
	\begin{subfigure}[t]{0.4\linewidth}
		\caption{)}
	%	\def\realCaptionA{#2}
		\centering
		\includegraphics[width = 1.0\linewidth]{#1}
		\label{fig:#3}		
	\end{subfigure}
	\quad
	\begin{subfigure}[t]{0.4\textwidth}
		\caption{)}
	%	\def\realCaptionB{#5}
		\centering
		\includegraphics[width = 1.0\linewidth]{#4}
		\label{fig:#6}
	\end{subfigure}
	%\def\realCaptionAll{#7}
	\def\showncap {#7\\ a) #2, b) #5}
	\caption[#7]{\showncap}
	\label{fig:#8}
\end{figure}
}

%---------------------------------------------------------------------------------
% Change table style

\captionsetup[table]{
    listformat=empty,
    name=Tabela,
    singlelinecheck=off, 
    justification=raggedright, 
    labelsep=period,
    position=above,
    aboveskip=0pt, 
	belowskip=-4pt,
    width=\linewidth,
    labelfont={small, bf},
    font={small}    
    }

%---------------------------------------------------------------------------------

\makeatletter

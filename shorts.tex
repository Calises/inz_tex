
%\clearpage
\addcontentsline{toc}{chapter}{Wykaz ważniejszych oznaczeń i skrótów}
\chapter*{Wykaz ważniejszych oznaczeń i skrótów}

%Sortowanie automatyczne, osobno dla skrótów i oznaczeń matematycznych
%Można dodawać w dowolnej kolejności oznaczenia - algorytm sobie z tym poradzi
%Użycie:
%\begin{sortedlist}
% \mathitem{oznaczenie_matematyczne}{wyjaśnienie oznaczenia}
% \shortitem{skrót}{wyjaśnienie skrótu}
%\end{sortedlist}

%Przykład!
\begin{sortedlist}
  \mathitem{$f$}{częstotliwość [Hz]}
  \mathitem{$k$}{współczynnik wypełnienia [\%]}
  \mathitem{$F$}{siła [N]}
  \mathitem{$B$}{indukcja magnetyczna [T]}
  \mathitem{$I$}{natężenie prądu [A]}
  \mathitem{$L$}{długość przewodu [m]}
  \mathitem{$dT$}{czas trwania niezerowej wartości sygnału [s]}
  \mathitem{$G(s)$}{transmitancja}
  \mathitem{$T$}{okres sygnału [s]}
  \mathitem{$R$}{rezystancja [$\Omega$]}
  \mathitem{$U$}{napięcie [V]}
  \shortitem{AVR}{8-bitowe mikrokontrolery firmy Atmel}
  \shortitem{HDMI}{High Definition Multimedia Interface (multimedialny interfejs wysokiej rozdzielczości)}
  \shortitem{CSI}{Camera Serial Interface (interfejs szeregowy kamery)}
  \shortitem{IP}{Internet Protocol (protokół internetowy)}
  \shortitem{GPIO}{General Purpose Input/Output (wejścia/wyjścia ogólnego przeznaczenia)}
  \shortitem{CLK}{Clock (zegar)}
  \shortitem{ICP}{Input Capture 1 (wejście przechwycenia)}
  \shortitem{ICR}{Input Capture Register 1 (rejestr przechwycenia)}
  \shortitem{DDR}{Data Direction Register (rejestr kierunku danych)}
  \shortitem{ICNC}{Input Capture Noise Canceler (usuwanie szumów wejścia przechwytującego)}
  \shortitem{ICES}{Input Capture Edge Select (wybór zbocza sygnału przechwytującego)}
  \shortitem{CS}{Clock Select (wybór zegara/preskaera)}
  \shortitem{ICIE}{Input Capture Interrupt Enable (włączenie przerwania wejścia przechwytującego)}
  \shortitem{LSB}{Least Significant Bit (najmniej znaczący bit)}
  \shortitem{MISO}{Master Input Slave Output (wejście nadrzędne, wyjście podrzędne)}
  \end{sortedlist}
  
  \begin{sortedlist}
  \mathitem{MOSI}{Master Output Slave Input (wyjście nadrzędne, wejście podrzędne)}
  \shortitem{MSB}{Most Significant Bit (najbardziej znaczący bit)}
  \shortitem{RISC}{Reduced Instruction Set Computing (architektura o ograniczonej liczbie rozkazów)}
  \shortitem{SPI}{Serial Peripheral Interface (peryferyjny interfejs szeregowy)}
  \shortitem{PWM}{Pulse-Width Modulation (regulacja przy pomocy wypełnienia)}
  \shortitem{TWI, I$^2$C}{2-wire Serial Interface (2-przewodowy interfejs szeregowy)}
  \shortitem{SSH}{Secure Shell (protokuł komunikacyjny)}
  \shortitem{WLAN}{Wireless Local Area Network (bezprzewodowa sieć lokalna)}
  \shortitem{SDA}{Serial Data Line (szeregowa linia danych)}
  \shortitem{SCL}{Serial Clock Line (szeregowa linia zegarowa)}
  \shortitem{SCK}{Serial Clock (szeregowy zegar taktujący)}
  \shortitem{SS}{Slave Select (wybór układu podrzędnego)}
  \shortitem{TCNT}{Timer Counter 1 (licznik układu czasowego)}
  \shortitem{TCCR}{Timer Counter Control Register (rejestr kontrolny licznika)}
  \shortitem{TOIE}{Timer Overflow Interrupt Enable (włączenie przerwania przepełnienia licznika)}
  \shortitem{OCR}{Output Compare Register (wyjściowy rejestr porównujący)}
  \shortitem{OC}{Output Compare (wyjściowy pin PWM)}
  \shortitem{WGM}{Waveform Generation Mode (tryb generatora funkcyjnego)}
  \shortitem{TWCR}{TWI Control Register (rejestr kontrolny TWI)}
  \shortitem{TWIE}{TWI Interrupt Enable (uruchomienie przerwania TWI)}
  \shortitem{TWEN}{TWI Enable (uruchomienie TWI)}
  \shortitem{TWEA}{TWI Enable Acknowlage (uruchomienie obsługi bitu potwierdzenia)}
  \shortitem{TWAR}{TWI Address Register (rejestr adresu urządzenia)}
  \shortitem{TWDR}{TWI Shift Data Register (przesuwny rejestr danych TWI)}
  \shortitem{OC}{Output Compare (zewnętrzne porównanie)}
  \shortitem{USB}{Universal Serial Bus (uniwersalna magistrala szeregowa)}
\end{sortedlist}

\begin{thebibliography}{9}
% Dodajemy bibliografi� do spisu tre�ci
\addcontentsline{toc}{chapter}{Bibliografia}
\bibitem{SJP}
\emph{S�ownik j�zyka polskiego}, http://www.sjp.pwn.pl/, (data dost�pu 20.10.2015 r.)
\bibitem{Czech}
\textsc{Leszek Rafi�ski, Aleksandra Bobcow, Andrzej Grono}: \emph{Roboty mobilne z autonomiczn� nawigacj� - stan obecny i perspektywy na najbli�sze lata}. Gda�sk, 2007  
\bibitem{def_robota}
\emph{Teoria robotyki}, http://www.robotyka.com/, (data dost�pu 21.10.2015 r.)
\bibitem{prawa_robota}
\emph{Prawa robotyki i bunt robot�w}, http://www.kopernik.org.pl/, (data dost�pu 11.10.2015 r.)
\bibitem{da_vinci}
\textsc{Eugeniusz Olszewski}: \emph{Leonardo da Vinci jako prekursor nauk technicznych}, 1969  
\bibitem{Redlarski}
\textsc{Grzegorz Redlarski, Andrzej Grono, Mariusz D�bkowski}: \emph{Perspektywy rozwoju robotyki}. Gda�sk, 2004  


\end{thebibliography}


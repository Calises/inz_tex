\begin{thebibliography}{9}
% Dodajemy bibliografię do spisu treści
\addcontentsline{toc}{chapter}{Bibliografia}
\bibitem{SJP}
\emph{Słownik języka polskiego}, http://sjp.pwn.pl/, (data dostępu 20.10.2015 r.)

\bibitem{Czech}
\textsc{Leszek Rafiński, Aleksandra Bobcow, Andrzej Grono}: \emph{Roboty mobilne z autonomiczną nawigacją - stan obecny i perspektywy na najbliższe lata}. Gdańsk, 2007 r.  

\bibitem{def_robota}
\emph{Teoria robotyki}, http://robotyka.com/, (data dostępu 21.10.2015~r.)

\bibitem{prawa_robota}
\emph{Prawa robotyki i bunt robotów}, http://kopernik.org.pl/, (data dostępu 11.10.2015~r.)

\bibitem{da_vinci}
\textsc{Eugeniusz Olszewski}: \emph{Leonardo da Vinci jako prekursor nauk technicznych}, 1969~r.

\bibitem{robot_squee}
\emph{Squee: The Robot Squirrel}, http://computerhistory.org/, (data dostępu 22.10.2015~r.)

\bibitem{wojna_pancerna}
\textsc{Tim Ripley}: \emph{Wojna pancerna. Strategia i taktyka}. Warszawa, 2008~r.

\bibitem{programy_rozwoju}
\textsc{Wojciech Zajler, Marek Ł. Grabania}: \emph{Perspektywiczne programy rozwoju pojazdów gąsienicowych}. „Szybkobieżne Pojazdy Gąsienicowe”, nr 2, 2003~r.

\bibitem{czolg_przyszlosci}
\textsc{Marek Dąbrowski}: \emph{Czołg – obecnie i w przyszłości}. „Szybkobieżne Pojazdy Gąsienicowe” nr 2, 2011~r.

\bibitem{kierunek_rozwoju}
\textsc{Wojciech Zajler, Marek Ł. Grabania}: \emph{Koncepcja modułowego specjalnego pojazdu wielozadaniowego}. „Szybkobieżne Pojazdy Gąsienicowe” nr 1, 2004~r.

\bibitem{Redlarski}
\textsc{Grzegorz Redlarski, Andrzej Grono, Mariusz Dąbkowski}: \emph{Perspektywy rozwoju robotyki}. Gdańsk, 2004~r.

\bibitem{Tadeusiewicz}
\textsc{Ryszard Tadeusiewicz, Przemysław Korohoda}: \emph{Komputerowa analiza i przetwarzanie obrazów}. Kraków 1997~r.

\bibitem{Malina}
\textsc{Witlod Malina, Maciej Smiatacz}: \emph{Cyfrowe przetwarzanie obrazów}. Warszawa 2008~r.

\bibitem{Tadeusiewicz_flasinski}
\textsc{Ryszard Tadeusiewicz, Mariusz Flasiński}: \emph{Rozpoznawanie obrazów}. Warszawa 1991~r.

\bibitem{Materka}
\textsc{Andrzej Materka, Paweł Strumiłło}: \emph{Wstęp do komputerowej analizy obrazów}. Łódzka 2009~r.

\end{thebibliography}

\begin{thebibliography}{9}
% Dodajemy bibliografię do spisu treści
%\addcontentsline{toc}{chapter}{Wykaz literatury}

\bibitem{Czech}
Rafiński L., Bobcow A., Grono A.: \emph{Roboty mobilne z autonomiczną nawigacją - stan obecny i perspektywy na najbliższe lata}. Gdańsk, 2007 r.  

\bibitem{def_robota}
Teoria robotyki, http://robotyka.com/, (data dostępu 21.10.2015~r.)

\bibitem{prawa_robota}
Prawa robotyki i bunt robotów, http://kopernik.org.pl/, (data dostępu 11.10.2015~r.)

\bibitem{da_vinci}
Olszewski E.: \emph{Leonardo da Vinci jako prekursor nauk technicznych}, 1969~r.

\bibitem{robot_squee}
Squee: The Robot Squirrel, http://computerhistory.org/, (data dostępu 22.10.2015~r.)

\bibitem{Malina}
Malina W., Smiatacz M.: \emph{Cyfrowe przetwarzanie obrazów}. Warszawa 2008~r.

\bibitem{Tadeusiewicz_flasinski}
Tadeusiewicz R., Flasiński M.: \emph{Rozpoznawanie obrazów}. Warszawa 1991~r.

\bibitem{Materka}
Materka A., Strumiłło P.: \emph{Wstęp do komputerowej analizy obrazów}. Łódzka 2009~r.

\bibitem{nota}
\emph{8-bit Atmel Microcontroller with 4/8/16K Bytes In-System Programmable Flash}, Atmel.

\bibitem{Sonka}
Sonka M., Hlavac V., Boyle R.: \emph{Image Processing, Analysis, and Machine Vision}. Pacific Grove, 1999~r.

\bibitem{Yuen}
Yuen HK, Princen J., Illingworth J., Kittler J.: \emph{Comparative study of hough transform methods for circle finding.}. "Image and Vision Computing" nr 8(1), 1990~r.

\end{thebibliography}

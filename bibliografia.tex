\begin{thebibliography}{9}
% Dodajemy bibliografi� do spisu tre�ci
\addcontentsline{toc}{chapter}{Bibliografia}
\bibitem{SJP}
\emph{S�ownik j�zyka polskiego}, http://sjp.pwn.pl/, (data dost�pu 20.10.2015 r.)

\bibitem{Czech}
\textsc{Leszek Rafi�ski, Aleksandra Bobcow, Andrzej Grono}: \emph{Roboty mobilne z autonomiczn� nawigacj� - stan obecny i perspektywy na najbli�sze lata}. Gda�sk, 2007 r.  

\bibitem{def_robota}
\emph{Teoria robotyki}, http://robotyka.com/, (data dost�pu 21.10.2015~r.)

\bibitem{prawa_robota}
\emph{Prawa robotyki i bunt robot�w}, http://kopernik.org.pl/, (data dost�pu 11.10.2015~r.)

\bibitem{da_vinci}
\textsc{Eugeniusz Olszewski}: \emph{Leonardo da Vinci jako prekursor nauk technicznych}, 1969~r.

\bibitem{robot_squee}
\emph{Squee: The Robot Squirrel}, http://computerhistory.org/, (data dost�pu 22.10.2015~r.)

\bibitem{wojna_pancerna}
\textsc{Tim Ripley}: \emph{Wojna pancerna. Strategia i taktyka}. Warszawa, 2008~r.

\bibitem{programy_rozwoju}
\textsc{Wojciech Zajler, Marek �. Grabania}: \emph{Perspektywiczne programy rozwoju pojazd�w g�sienicowych}. �Szybkobie�ne Pojazdy G�sienicowe�, nr 2, 2003~r.

\bibitem{czolg_przyszlosci}
\textsc{Marek D�browski}: \emph{Czo�g � obecnie i w przysz�o�ci}. �Szybkobie�ne Pojazdy G�sienicowe� nr 2, 2011~r.

\bibitem{Redlarski}
\textsc{Grzegorz Redlarski, Andrzej Grono, Mariusz D�bkowski}: \emph{Perspektywy rozwoju robotyki}. Gda�sk, 2004~r.

\bibitem{Tadeusiewicz}
\textsc{Ryszard Tadeusiewicz, Przemys�aw Korohoda}: \emph{Komputerowa analiza i przetwarzanie obraz�w}. Krak�w 1997~r.

\bibitem{Malina}
\textsc{Witlod Malina, Maciej Smiatacz}: \emph{Cyfrowe przetwarzanie obraz�w}. Warszawa 2008~r.

\bibitem{Tadeusiewicz_flasinski}
\textsc{Ryszard Tadeusiewicz, Mariusz Flasi�ski}: \emph{Rozpoznawanie obraz�w}. Warszawa 1991~r.


\end{thebibliography}

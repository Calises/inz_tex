\namedchapter[Daniel Łukwiński]{Algorytm sterujący}

Omówiona w poprzednim rozdziale analiza obrazu dążąca do detekcji czerwonych okręgów jest pierwszy etapem kodu wykonującego się w pętli sterującej programu. Drugim etapem jest wykorzystanie wyników tej detekcji do sterowania całym czołgiem. Ogólny graf obrazujący działanie kodu sterującego robotem przedstawia ilustracja \ref{graf}. Punktem wyjścia dla algorytmu sterującego jest wynik detekcji okręgów przeprowadzonej za pomocą transformacji Haougha, z kolei końcowym etapem jest zawsze koniec aktualnej iteracji pętli, a więc i rozpoczęcie kolejnej, zaczynającej się od pobrania nowego obrazu i jego analizy.

% Define block styles
\tikzstyle{decision} = [diamond, draw, fill=white!20, 
    text width=6em, text badly centered, node distance=3cm, inner sep=1pt]
\tikzstyle{block} = [rectangle, draw, fill=white!20, 
    text width=8em, text centered, node distance=3cm, minimum height=3em]
\tikzstyle{line} = [draw, -latex']
\tikzstyle{cloud} = [draw, ellipse,fill=white!20, node distance=4cm,
    minimum width=8em, minimum height=4em]
    
\begin{figure}[H]
\begin{center}
\begin{tikzpicture}[node distance = 2cm, auto]
    \node [block] (init) {Analiza obrazu};
    \node [decision, below of=init] (wykrycie) {Wykryto okrąg?};
    \node [block, below left of=wykrycie] (przeszukiwanie) {Przeszukiwanie};
    \node [decision, below right of=wykrycie] (osKamery) {Kamera wycentrowana?};
    \node [block, below left of=osKamery] (centrowanieKamery) {Centrowanie kamery};
    \node [decision, below right of=osKamery] (osKandluba) {Kadłub wycentrowany?};
    \node [block, below left of=osKandluba] (centrowanieKadluba) {Centrowanie kadłuba};
    \node [decision, below right of=osKandluba] (rozmiar) {Rozmiar koła};
    \node [block, below left of=rozmiar] (jazdaTyl) {Jazda w tył};
    \node [block, below right of=rozmiar] (jazdaPrzod) {Jazda w przód};
   	\node [cloud, below of=rozmiar] (koniec) {Koniec pętli};
    
    \path [line] (init) -- (wykrycie);
    \path [line] (wykrycie) -- node [midway, above, sloped] {Nie} (przeszukiwanie);
    \path [line] (wykrycie) -- node [very near start, sloped] {Tak} (osKamery);
    \path [line] (osKamery) -- node [midway, above, sloped] {Nie} (centrowanieKamery);
    \path [line] (osKamery) -- node [very near start, sloped] {Tak} (osKandluba);
    \path [line] (osKandluba) -- node [midway, above, sloped] {Nie} (centrowanieKadluba);
    \path [line] (osKandluba) -- node [very near start, sloped] {Tak} (rozmiar);
    \path [line] (rozmiar) -- node [midway, above, sloped] {Za duże} (jazdaTyl);
    \path [line] (rozmiar) -- node [at start, sloped] {Za małe} (jazdaPrzod);    
    \path [line] (rozmiar) -- node [midway, above, sloped] {Odpowiednie} (koniec);
    
    \path [line] (przeszukiwanie) |- (koniec);
    \path [line] (centrowanieKamery) |- (koniec);
    \path [line] (centrowanieKadluba) |- (koniec);
    \path [line] (jazdaTyl) |- (koniec);
    \path [line] (jazdaPrzod) |- (koniec);
\end{tikzpicture}
\caption[Graf algorytmu sterującego.]\small{Ilustracja przedstawia graf algorytmu sterowania robotem.}
\label{graf}
\end{center}
\end{figure}

\namedsection{Przeszukiwanie}
Pierwszym rozgałęzieniem na grafie jest sprawdzenie, czy na obrazie wykryty został okrąg. Dokonywane jest to poprzez sprawdzenie długości wektora, w którym zapisywane są parametry znalezionych okręgów. W przypadku jego zerowej długości algorytm przechodzi do sekcji kodu odpowiedzialnej za przeszukiwanie otoczenia.

Przeszukiwanie ma za zadanie przeanalizować pod kątem obecności celu całe otoczenie robota. Stosowana kamera umożliwia rejestrację obrazu w zakresie $41,4^\circ$ w poziomie oraz $53,5^\circ$ w pionie. Ponadto należy uwzględnić fakt, że wieża posiada możliwość obrotu w zakresie $180^\circ$. W celu rejestracji całego otoczenia robota zdecydowano się więc, że najpierw przeszukana będzie przestrzeń możliwa do rejestracji w jednym położeniu korpusu czołgu, a w przypadku braku efektów cały pojazd zostanie obrócony o $180^\circ$, aby móc przeanalizować otoczenie znajdujące się wcześniej poza zasięgiem kamery na wieży. 

Realizacja powyższego algorytmu polega na podzieleniu go na sześć etapów. Każde żądanie przeszukania, przedstawione na grafie, polega na wykonaniu jednego z tych etapów. Pierwszych pięć polega na kolejnym ustawianiu wieży w pięciu położeniach umożliwiających obserwację dostępnego otoczenia. Pierwsza pozycja wychyla wierzę o $72^\circ$ w lewo względem osi kadłuba, a każdy kolejny etap zakłada obrócenie wieży o $36^\circ$ w prawo. Kolejne ustawienia we wszystkich etapach przedstawia tabela \ref{tab:przeszukiwanie}.
\begin{table}[h!tb]
\centering
\small
\caption{Pięć etapów przeszukiwania}
\begin{tabularx}{\linewidth}[c]{|l|X|X|X|X|X|} 
\hline
	Etap przeszukiwania & 1 & 2 & 3 & 4 & 5 \\ \hline
	Kąt względem osi kadłuba & $-72^\circ$ & $-36^\circ$ & $0^\circ$ & $36^\circ$ & $72^\circ$ \\ \hline
 	\noalign{\smallskip}
\end{tabularx}
\label{tab:przeszukiwanie}
\vspace{-8pt}
\end{table}
Ostatni, szósty etap przeszukiwania polega na powrocie wieży do pozycji pierwszej, oraz obróceniu całego pojazdu o $180^\circ$. W przypadku kolejnego żądania przeszukania algorytm zaczyna ponownie od pierwszego etapu.

Realizacja programowa przedstawionych zasad opiera się na przechowywaniu w zmiennej globalnej informacji o ostatnim wykonanym etapie przeszukiwania. Wywołanie funkcji przeszukującej polega więc w pierwszej kolejności na iteracji tej zmiennej, a następnie, za pomocą instrukcji \texttt{switch}, przejściu do fragmentu kodu ustawiającego wieżę (lub wieżę i kadłub) w żądanej pozycji.

Do ustawiania wieży służy, omawiana we wcześniejszej części pracy, biblioteka \textit{WiringPi}. Dokonywana na początku pracy programu inicjalizacja przygotowuje Raspberry do zmiany wypełnienia generowanego sygnału PWM w dowolnym miejscu kodu.  Służy do tego funkcja \texttt{void pwmWrite(int pin, int value);}, w której zmienna \texttt{pin} oznacza numer portu na szynie GPIO Raspberry (w przypadku sygnałów PWM jest to port 18 oraz 19), a wartość \texttt{value} definiuje poziom wypełnienia sygnału. Wykonanie obrotu czołgu z kolei wymaga wysłania odpowiednich sygnałów sterujących do mikrokontrolerów ATmega, odpowiedzialnych za sterowanie mostkami H przypisanymi do silników. Komunikacja odbywa się poprzez magistralę TWI, obsługiwaną także za pomocą biblioteki \textit{WiringPi}. Do nawiązania połączenia z urządzeniem po tej magistrali służy funkcja \texttt{int wiringPiI2CSetup(int devId);}, w której wartość \texttt{devId} oznacza adres urządzenia na magistrali TWI, a zwracana wartość typu \texttt{int} jest standardowym deskryptorem (identyfikatorem) pliku, stosowanym w systemie Linux, który w tym przypadku wskazuje na adresowane urządzenie. Parametr ten jest wykorzystywany w kolejnej funkcji, służącej bezpośrednio do wysyłania danych po magistrali. Jej deklaracja przedstawia się następująco: \texttt{int wiringPiI2CWriteReg8(int fd, int data);}. Argument \texttt{fd} jest wspomnianym wcześniej deskryptorem identyfikującym urządzenie w systemie Linux, z kolei \texttt{data}
zawierać ma przesyłane do urządzenia dane. W omawianym programie do każdego z mikrokontrolerów jako dane przesyłane są dwie liczby 8-bitowe: pierwsza z nich odpowiada za prędkość poruszania się silnika w przód, druga prędkość poruszania się silnika w tył. Wysłane w ten sposób ustawienia powodują rozpoczęcie generowania przez ATmegi odpowiednich wypełnień sygnałów PWM, decydujących o prędkości obrotowej silników napędzających robota. W celu zatrzymania ruchu należy ponownie wysłać do obu urządzeń dane sterujące, tym razem wypełnione zerami. Obrót korpusu pojazdu o $180^\circ$ polega więc na wysłaniu do mikrontrolerów polecenia obrotu jednego silnika w przód, a drugiego w tył z taką samą prędkością, a następnie, po czasie koniecznym dla wykonania żądanego obrotu, zatrzymania pracy obu silników.



\namedsection{Centrowanie kamery i kadłuba}

\namedsection{Jazda}


















    

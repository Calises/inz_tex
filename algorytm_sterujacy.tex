\namedchapter[Daniel Łukwiński]{Algorytm sterujący}
Postawionym przed robotem zadaniem jest, poza znalezieniem celu, odpowiednie zachowanie się po jego wykryciu. Jako cel przyjęto ustawienie się przodem do celu, w odpowiedniej od niego odległości. Pętla sterująca robotem podzielona została na dwie części. Pierwsza z nich, omówiona w poprzednim rozdziale, polegała na przetwarzaniu i analizie obrazu z kamery w celu wykrycia czerwonych kół znajdujących się w otoczeniu robota. Uzyskane w ten sposób informacje wykorzystywane są w drugiej części pętli - pętli sterującej ruchem robota. Ogólny graf obrazujący działanie algorytmu sterującego robotem przedstawia ilustracja \ref{graf}. Punktem wyjścia dla algorytmu sterującego jest wynik detekcji okręgów przeprowadzonej za pomocą transformacji Hougha.

% Define block styles
\tikzstyle{decision} = [diamond, draw, fill=white!20, 
    text width=6em, text badly centered, node distance=3cm, inner sep=1pt]
\tikzstyle{block} = [rectangle, draw, fill=white!20, 
    text width=8em, text centered, node distance=3cm, minimum height=3em]
\tikzstyle{line} = [draw, -latex']
\tikzstyle{cloud} = [draw, ellipse,fill=white!20, node distance=4cm,
    minimum width=8em, minimum height=4em]
    
\begin{figure}[H]
\begin{center}
\begin{tikzpicture}[node distance = 2cm, auto]
    \node [block] (init) {Analiza obrazu};
    \node [decision, below of=init] (wykrycie) {Wykryto okrąg?};
    \node [block, below left of=wykrycie] (przeszukiwanie) {Przeszukiwanie};
    \node [decision, below right of=wykrycie] (osKamery) {Kamera wycentrowana?};
    \node [block, below left of=osKamery] (centrowanieKamery) {Centrowanie kamery};
    \node [decision, below right of=osKamery] (osKandluba) {Kadłub wycentrowany?};
    \node [block, below left of=osKandluba] (centrowanieKadluba) {Centrowanie kadłuba};
    \node [decision, below right of=osKandluba] (rozmiar) {Rozmiar koła};
    \node [block, below left of=rozmiar] (jazdaTyl) {Jazda w tył};
    \node [block, below right of=rozmiar] (jazdaPrzod) {Jazda w przód};
   	\node [cloud, below of=rozmiar] (koniec) {Koniec pętli};
    
    \path [line] (init) -- (wykrycie);
    \path [line] (wykrycie) -- node [midway, above, sloped] {Nie} (przeszukiwanie);
    \path [line] (wykrycie) -- node [very near start, sloped] {Tak} (osKamery);
    \path [line] (osKamery) -- node [midway, above, sloped] {Nie} (centrowanieKamery);
    \path [line] (osKamery) -- node [very near start, sloped] {Tak} (osKandluba);
    \path [line] (osKandluba) -- node [midway, above, sloped] {Nie} (centrowanieKadluba);
    \path [line] (osKandluba) -- node [very near start, sloped] {Tak} (rozmiar);
    \path [line] (rozmiar) -- node [midway, above, sloped] {Za duże} (jazdaTyl);
    \path [line] (rozmiar) -- node [at start, sloped] {Za małe} (jazdaPrzod);    
    \path [line] (rozmiar) -- node [midway, above, sloped] {Odpowiednie} (koniec);
    
    \path [line] (przeszukiwanie) |- (koniec);
    \path [line] (centrowanieKamery) |- (koniec);
    \path [line] (centrowanieKadluba) |- (koniec);
    \path [line] (jazdaTyl) |- (koniec);
    \path [line] (jazdaPrzod) |- (koniec);
\end{tikzpicture}
\caption[Graf algorytmu sterującego.]\small{Ilustracja przedstawia graf algorytmu sterowania robotem.}
\label{graf}
\end{center}
\end{figure}

\namedsection{Algorytm szukania celu}
Pierwszym zadaniem algorytmu sterującego robotem jest sprawdzenie, czy za pomocą analizy obrazu z kamery wykryte zostało czerwone koło. Dokonywane jest to poprzez sprawdzenie długości wektora, w którym zapisywane są parametry znalezionych okręgów. W przypadku jego zerowej długości algorytm przechodzi do sekcji kodu odpowiedzialnej za przeszukiwanie otoczenia.

Przeszukiwanie ma za zadanie przeanalizować pod kątem obecności celu całe otoczenie robota. Stosowana kamera umożliwia rejestrację obrazu w zakresie $41,4^\circ$ w poziomie oraz $53,5^\circ$ w pionie. Ponadto należy uwzględnić fakt, że wieża posiada możliwość obrotu w zakresie $180^\circ$, a więc całkowity zasięg rejestracji obrazu w poziomie bez poruszania korpusem wynosi $180^\circ + 2 * (\frac{1}{2} \cdot 41,4^\circ) = 221,4^\circ$. W celu rejestracji całego otoczenia robota zdecydowano się więc, że najpierw przeszukana będzie przestrzeń możliwa do rejestracji w jednym położeniu korpusu robota, a w przypadku braku efektów cały pojazd zostanie obrócony o $180^\circ$, aby móc przeanalizować otoczenie znajdujące się wcześniej poza zasięgiem kamery na wieży.

Realizacja powyższych założeń polega na podzieleniu algorytmu na sześć etapów. Za każdym razem, gdy na obrazie nie zostanie znaleziony żaden okrąg, powstaje potrzeba zmiany położenia kamery poprzez realizację jednego, kolejnego etapu.
Pięć pierwszych etapów polega na kolejnym ustawianiu wieży w pięciu położeniach, równomiernie rozłożonych w jej obszarze ruchu i umożliwiających obserwację dostępnego otoczenia. Pierwsza pozycja wychyla wieżę o $72^\circ$ w lewo względem osi kadłuba, a każdy kolejny etap zakłada obrócenie wieży o $36^\circ$ w prawo. Kolejne ustawienia we wszystkich etapach przedstawia tabela \ref{tab:przeszukiwanie}. Odległość między dwoma skrajnymi położeniami wieży z kamerą wynosi $144^\circ$, co w połączeniu z kątem pracy kamery daje obszar o zakresie $185,5^\circ$, co z $5,5^\circ$ zapasem spełnia postawione założenie.
\begin{table}[h!tb]
\centering
\small
\caption{Pięć etapów przeszukiwania}
\begin{tabularx}{\linewidth}[c]{|l|X|X|X|X|X|} 
\hline
	Etap przeszukiwania & 1 & 2 & 3 & 4 & 5 \\ \hline
	Kąt względem osi kadłuba & $-72^\circ$ & $-36^\circ$ & $0^\circ$ & $36^\circ$ & $72^\circ$ \\ \hline
 	\noalign{\smallskip}
\end{tabularx}
\label{tab:przeszukiwanie}
\vspace{-8pt}
\end{table}
Ostatni, szósty, etap przeszukiwania polega na powrocie wieży do pozycji pierwszej oraz obróceniu całego pojazdu o $180^\circ$. W przypadku kolejnego żądania przeszukania algorytm zaczyna ponownie od pierwszego etapu, dzięki czemu w przypadku ciągłego braku celu w polu widzenia kamery całe otoczenie robota jest cyklicznie przeszukiwane.

Realizacja programowa przedstawionych zasad opiera się na przechowywaniu w pamięci informacji o ostatnim wykonanym etapie przeszukiwania. Wywołanie funkcji przeszukującej polega więc w pierwszej kolejności na iteracji tej zmiennej, a następnie, za pomocą instrukcji \texttt{switch}, przejściu do fragmentu kodu ustawiającego wieżę (lub wieżę i kadłub) w żądanej pozycji.

Do ustawiania wieży służy, omawiana we wcześniejszej części pracy, biblioteka \textit{WiringPi}. Dokonywana na początku pracy programu inicjalizacja przygotowuje \textit{Raspberry Pi} do zmiany wypełnienia generowanego sygnału PWM w dowolnym miejscu kodu.  Służy do tego funkcja \texttt{pwmWrite}, do której przekazywany jest numer portu na którym ma być generowany sygnał oraz wartość definiującą poziom wypełnienia sygnału PWM, przekładający się bezpośrednio na wychylenie serwomechanizmu.

Wykonanie obrotu robota z kolei wymaga wysłania odpowiednich sygnałów sterujących do mikrokontrolerów ATmega, odpowiedzialnych za sterowanie mostkami H przypisanymi do silników. Komunikacja odbywa się poprzez magistralę TWI i polega na przesłaniu do każdego z mikrokontrolerów dwóch liczb 8-bitowych: jednej odpowiedzialnej za prędkość poruszania się silnika w przód, drugiej za prędkość poruszania się silnika w tył. Chcąc wykonać obrót korpusu pojazdu o $180^\circ$ należy więc wysłać do mikrontrolerów polecenia obrotu jednego silnika w przód, a drugiego w tył z taką samą prędkością, a następnie, po czasie koniecznym dla wykonania żądanego obrotu, zatrzymania pracy obu silników poprzez wysłanie samych zer.

\namedsection{Algorytm centrowania kamery i kadłuba}
W przypadku wykrycia okręgu sprawdzane jest, czy wieża oraz kadłub znajdują się w odpowiednim ustawieniu względem wykrytego obiektu. W tym celu na podstawie informacji o środku najlepszego znalezionego okręgu (przez ,,najlepszy" rozumiany jest okrąg z najwyższą wartością w macierzy akumulatora transformaty Hougha) wyliczane jest odchylenie znalezionego obiektu od osi kamery. Łatwo można policzyć odległość w pikselach dowolnego punktu na obrazie od jego środka, co z kolei, przy znajomości kątów pracy kamery oraz rozdzielczości obrazu można przełożyć na odchylenie kątowe. W przypadku gdy odchylenie celu od osi kamery jest mniejsze niż $5^\circ$ w obu płaszczyznach uznaje się kamerę za wycentrowaną. W przeciwnym wypadku następuje centrowanie, korzystające z informacji o odchyleniu celu od osi kamery oraz z wiedzy o tym, w jakim stopniu należy zmienić wypełnienie sygnału PWM sterującego serwomechanizmami, aby odchyliły się one o $1^\circ$. Po centrowaniu kamery następuje analiza kolejnego zdjęcia.

W przypadku, gdy kamera została uznana za wycentrowaną podobne sprawdzenie dotyczy kadłuba. W tym wypadku korzysta się z wiedzy o tym, że kamera jest już wycentrowana. Sprawdza się więc (zachowując także margines $5^\circ$) czy wieża jest w położeniu współosiowym z kadłubem. Służy do tego proste sprawdzenie, czy serwomechanizm odpowiedzialny za obrót wieży w poziomie znajduje się w ustawieniu znanym jako środkowe. Jeśli tak, uznaje się, że także oś kadłuba jest wycentrowana na celu. Jednak w przypadku braku tej zbieżności następuje centrowanie kadłuba. Polega ono na ustawieniu wieży w położeniu środkowym (aby po centrowaniu kadłuba także ona pozostała nakierowana na cel) oraz obrocie całego pojazdu o kąt, równy wcześniejszemu odchyleniu osi wieży od osi kadłuba. Do obrotu robota wykorzystuje się stałą prędkość obrotową silnika, równą 1 Hz, oraz uzyskaną w trakcie testów wiedzę o tym, jaki czas muszą pracować silniki z tą prędkością aby obrócić całego robota o $1^\circ$. W przedstawianym przypadku do obrotu pojazdu o $1^\circ$ siniki muszą pracować przez okres 18,5 ms.

\namedsection{Algorytm podjeżdżania do wykrytego obiektu}
W sytuacji, gdy tak wieża, jak i kadłub zostają uznane za wycentrowane algorytm sterujący przechodzi do etapu, który ma za zadanie ustawić pojazd w optymalnej odległości od celu. Jako odległość optymalną przyjęto taką, w której cel będzie zajmował od $\frac{1}{2}$ do $\frac{2}{3}$ szerokości obrazu. Do sprawdzania tego warunku wykorzystywany jest promień okręgu, który, obok współrzędnych środka, jest zwracany przez funkcję wykonującą transformatę Hougha. Porównanie promienia okręgu z rozdzielczością obrazu z kamery daje informację o tym, czy robot znajduje się zbyt blisko celu i konieczny jest ruch w tył, czy też cel jest zbyt daleko i wymagana jest jazda w przód. Sama jazda realizowana jest poprzez uruchomienie obu silników z jednakową prędkością na czas 1 s. Po tym czasie silniki są zatrzymywane i algorytm rozpoczyna analizę nowego obraz z kamery.

















    

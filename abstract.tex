
\chapter*{Streszczenie}
Celem projektu inżynierskiego jest zaprojektowanie i wykonanie autonomicznego robota typu czołg. Pojazd oparty będzie o mikrokomputer Raspberry Pi 2 model B wraz z dedykowaną kamerą umieszczoną na wieżyczce. Na podstawie pozyskanego z niej obrazu wykrywane będzie znajdujące się w  pomieszczeniu czerwone koło. Zadaniem wieżyczki jest ustawienie kamery w taki sposób, aby znaleziony obiekt znajdował się w centrum kadru. Następnie cały pojazd ustawia się przodem do celu i podjeżdża do niego. Dodatkowo sterownik silników wykonany będzie na podstawie mikrokontrolerów Atmel AVR. Wpływ jakości oświetlenia na obraz kompensowany będzie przy wykorzystaniu algorytmu wykonującego odpowiednie operacje na poszczególnych pikselach. Na tak spreparowanym obrazie dokonywana będzie operacja transformacji Hougha. Dzięki niej na obrazie wyodrębniającym czerwone obiekty planowane jest znalezienie jedynie czerwonych kół. Przetwarzanie i analiza obrazu oparta będzie o funkcjonalności oferowane przez bibliotekę \textit{OpenCV}.
\\\\
\noindent
\textbf{Słowa kluczowe}: Raspberry Pi, OpenCV, robot mobilny, przetwarzanie obrazu, analiza obrazu
\\

\noindent
\textbf{Dziedzina nauki i techniki, zgodnie z wymogami OECD}: automatyka i robotyka
\chapter*{Abstract}
And here we type the abstract of our thesis on 1 page max.
\\\\
\noindent
\textbf{Keywords}:Raspberry Pi, OpenCV, mobile robot, image processing, image analysis
\namedchapter{Wstęp i cel pracy}
Celem projektu inżynierskiego jest budowa autonomicznego pojazdu mobilnego przypominającego czołg. Temat pracy jest bardzo otwarty, co pozwala na dużą swobodę dotyczącą wyboru założeń konstrukcyjnych. Robot zostanie wyposażony w obrotową wieżyczkę znajdującą się na korpusie pojazdu oraz zamocowanym do niej "działem" mogącym zmieniać swoje położenie. Urządzenie w sposób autonomiczny stara się zlokalizować cel oraz określić jego położenie względem samego siebie. W ramach projektu należy: zbudować robota, wykonać projekt elektroniki, zaimplementować algorytm sterujący robotem, przeprowadzić niezbędne testy działania systemu oraz przygotować dokumentację techniczną.

Koncepcja inteligentnych robotów bojowych znajduje się w kręgu zainteresowań służb specjalnych takich jak wojsko czy policja. Pozwalają one przeprowadzać wiele niebezpiecznych operacji bez narażania życia załogi. Podczas prowadzenia wojen największy nacisk kładzie się na ochronę żołnierzy, z tego względu, że wyszkolenie i zaadaptowanie nowej osoby na miejsce doświadczonej i wykwalifikowanej jednostki zajmuje zbyt wiele czasu oraz jest zanadto kosztowne. 

Autonomiczne pojazdy bojowe mogą docierać do miejsc, które są niedostępne dla ludzi oraz zdalnie sterowanych robotów. Często posyłane są w miejsca, w których komunikacja radiowa czy przewodowa z robotem jest niemożliwa bądź niestabilna, np. podczas prac prowadzonych w jaskiniach, zawalonych budynkach, tunelach bądź pod wodą. Nawet sterowane przez człowieka maszyny, wykonujące zadania w trudnych warunkach, powinny mieć zaimplementowane proste algorytmy, które przypadku utraty łączności, pozwalają na zachowanie aktualnej pozycji i podjęcie prób wznowienia komunikacji.

Pojazdy te biorą czynny udział w licznych badaniach naukowych oraz operacjach ratunkowych - nie tylko wojskowych, ale także medycznych. Zadania, które są przed nimi postawione są niejednokrotnie bardzo odpowiedzialne, gdyż to od nich może zależeć ludzkie życie. W związku z tym nie mogą popełniać błędów związanych z błędną interpretacją odebranych z czujników informacji. Podczas tworzenia projektu bardzo duży nacisk będzie nałożony na algorytm przetwarzanie obrazu otoczenia aby możliwie skutecznie oraz jednoznacznie rozpoznać cel.

Wybór tego tematu podyktowany był przede wszystkim chęcią zbudowania robota. Moim zdaniem jest to zwieńczenie całej zdobytej podczas trwania studiów wiedzy. Proces budowy autonomicznego pojazdu składający się zarówno z zaprojektowania części mechanicznej jak i programowej, jest bardzo wymagający i złożony. Wymaga on od konstruktora nie tylko wiedzy teoretycznej dotyczącej zasady budowy i działania poszczególnych podzespołów, ale także opanowania technik związanych z sterowaniem oraz komunikacją pomiędzy różnego typu urządzeniami. Realizacja tego projektu pozwoli na sprzężenie wiedzy teoretycznej ze światem rzeczywistym oraz ocenę skuteczności zastosowanych rozwiązań.

Autonomiczne roboty mobilne najczęściej poruszają się w nie do końca znanym im środowisku. Co za tym idzie - muszą być wyposażone w system nawigacyjny, przez który rozumiany jest zespół czujników pełniących funkcję sprzężenia zwrotnego z otaczającego pojazd świata. W naszym przypadku system głównie opierać się będzie o mikrokomputer wyposażony w kamerę video, który dodatkowo będzie wspierany kilkoma czujnikami odległości. Ich zadaniem będzie przede wszystkim wykrycie możliwych kolizji z przedmiotami znajdującymi się bezpośrednio przed robotem. Projekt oparty będzie o tzw. system wbudowany, czyli kompaktową, multimedialną jednostkę obliczeniową wykorzystywaną do realizacji specjalnych funkcji takich jak sterowanie pracą silnika czy analiza danych. Systemy wbudowane są (najczęściej) na stałe połączone z elementami wykonawczymi oraz pomiarowymi. Najważniejszą funkcją czołgu będzie przetwarzanie oraz analizowanie obrazów w czasie rzeczywistym, co wymaga stosunkowo dużej pamięci oraz mocy obliczeniowej. Powyższe wymagania pozwoliły wybrać platformę, na której zrealizowany będzie projekt, mianowicie \textit{Raspberry Pi} wraz z dedykowaną kamerą \textit{Raspberry Pi Camera Board}. 

Wybór modelu zawieszenia dla czołgu jest jednoznaczny - gąsienice. Rozwiązanie tego typu jest najczęściej spotykanym systemem jezdnym pojazdów wojskowych. Roboty bojowe często zmuszone są poruszać się w bardzo zróżnicowanym i nieprzyjaznym środowisku. Stawia to przed układem zawieszenia wiele wyzwań.  Gąsienice - dzięki równomiernemu rozłożeniu ciężaru (wiele pasywnych osi), dużej powierzchni styku z podłożem, prostocie sterowania oraz trwałości są niewątpliwie najlepszym rozwiązaniem w tej dziedzinie.

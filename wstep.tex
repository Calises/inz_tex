\namedchapter{Wstęp i cel pracy}
Celem projektu inżynierskiego jest budowa autonomicznego robota mobilnego przypominającego czołg. Robot zostanie wyposażony w obrotową wieżyczkę znajdującą się na korpusie pojazdu oraz zamocowanym do niej "działem" mogącym zmieniać swoje położenie. Robot w sposób autonomiczny stara się zlokalizować cel oraz określić jego położenie względem samego siebie. W ramach projektu należy: zbudować robota, wykonać projekt elektroniki, zaimplementować algorytm sterujący robotem, przeprowadzić niezbędne testy działania systemu oraz przygotować dokumentację techniczną.

Autonomiczne roboty mobilne najczęściej poruszają się w nie do końca znanym im środowisku. Co za tym idzie - muszą być wyposażone w system nawigacyjny, przez który rozumiany jest zespół czujników pełniących funkcję sprzężenia zwrotnego z otaczającego pojazd świata. W naszym przypadku system głównie opierać się będzie o mikrokomputer wyposażony w kamerę video, który dodatkowo będzie wspierany krańcówkami znajdującymi się na przedzie pojazdu. Ich zadaniem będzie przede wszystkim wykrycie możliwych kolizji z przedmiotami znajdującymi się bezpośrednio przed robotem. Projekt oparty będzie o tzw. system wbudowany, czyli kompaktową, multimedialną jednostkę obliczeniową wykorzystywaną do realizacji specjalnych funkcji takich jak sterowanie pracą silnika czy analiza danych. Systemy wbudowane są (najczęściej) na stałe połączone z elementami wykonawczymi oraz pomiarowymi. Najważniejszą funkcją robota będzie przetwarzanie oraz analizowanie obrazów w czasie rzeczywistym, co wymaga stosunkowo dużej pamięci oraz mocy obliczeniowej. Powyższe wymagania pozwoliły wybrać platformę, na której zrealizowany będzie projekt, mianowicie \textit{Raspberry Pi} wraz z dedykowaną kamerą \textit{Raspberry Pi Camera Board}. 

Wybór modelu zawieszenia dla robota jest jednoznaczny - gąsienice. Rozwiązanie tego typu jest najczęściej spotykanym systemem jezdnym pojazdów wojskowych. Roboty bojowe często zmuszone są poruszać się w bardzo zróżnicowanym i nieprzyjaznym środowisku. Stawia to przed układem zawieszenia wiele wyzwań.  Gąsienice - dzięki równomiernemu rozłożeniu ciężaru (wiele pasywnych osi), dużej powierzchni styku z podłożem, prostocie sterowania oraz trwałości są niewątpliwie najlepszym rozwiązaniem w tej dziedzinie.

\namedchapter{Podsumowanie}
Zakończenie każdego z przyjętych na wstępie etapów etapów kończyło się weryfikacją, czy spełnia on założenia projektowe. Tak więc w przypadku konstrukcji mechanicznej należało przede wszystkim sprawdzić jakość dopasowania poszczególnych elementów. Po wykonaniu płytki drukowanej w pierwszej kolejności sprawdzone zostały połączenia pomiędzy komponentami (na wypadek gdyby któraś ze ścieżek była przerwana, bądź element źle zlutowany) oraz pomierzyć napięcia w poszczególnych miejscach. Następnie możliwe było przejście do implementacji sterownika silników, którego wykonanie oceniane było poprzez analizę odpowiedzi skokowej silnika odczytanej przy pomocy mikrokontrolera oraz \textit{Raspberry Pi}. Kolejnym etapem było zapewnienie stałej łączności pomiędzy systemem wizyjnym oraz sterownikiem. W przypadku, gdy określoną sekwencję bitów udało się odczytać na Atmedze, podpunkt uznano za zrealizowany. Prace nad analizą obrazu można było prowadzić równolegle do pozostałych zadań. Wynika to z faktu, iż platformę \textit{Raspberry Pi} oraz moduł \textit{Raspberry Pi Camera HD} wystarczyło jedynie połączyć ze sobą oraz odpowiednio przygotować środowisko pracy. Program analizujący obraz testowany był na wielu różnorodnych zdjęciach. Gdy osiągnął zadowalający poziom - rozpoczęły się prace nad integracją wszystkich poszczególnych systemów. Wszystkie poprzednie prace wykonane były z dużą starannością więc połączenie ich ze sobą nie przyniosło większych problemów.

Ostatnim etapem były testy polegające na pozostawieniu w różnych miejscach pomieszczenia arkuszy formatu A4, na których znajdowały się czerwone koła o różnych rozmiarach, oraz obserwacja zachowania robota. Parametrami podlegającymi ocenie to: szybkość działania, dokładność sterowania oraz skuteczność wykrywania okręgu. W aspekcie szybkości działania udało się uzyskać czas analizy jednego zdjęcia pomiędzy 0,4 a 1,2 s. Jest to kompromis pomiędzy dokładnością pracy algorytmu a szybkością reakcji robota. Sygnały sterujące wysyłane średnio co 0,8 s wystarczają do sprawnego przeszukiwania pomieszczenia w celu znalezienia statycznego celu. Dokładność sterowania robotem osiągnęła więcej niż zadowalający poziom. Serwomechanizmy pozycjonujące wieżą mogą być ustawiane z precyzją poniżej $1^\circ$, zaś silniki napędzające pojazd mogą być sterowane oddzielnie, w bardzo szerokim zakresie prędkości. Implementacja sterownika silników umożliwiła także ustawienie równej prędkości obrotowej na obu z nich, dzięki której robot bez problemu porusza się po linii prostej mimo zastosowania osobnego napędu i całego układu sterującego dla prawej oraz lewej gąsienicy. Jedyne zastrzeżenia w trakcie testów budził efekt detekcji czerwonych kół. Mimo częstej poprawnej pracy nie cechował się on stuprocentową skutecznością. Założenie w postaci pracy przy zróżnicowanym oświetleniu wprowadziło dużą tolerancję dla intensywności dobieranej barwy czerwonej. Spowodowało to istotne ryzyko niewłaściwej interpretacji obrazu, którego wyeliminowanie z kolei znacząco wydłużyłoby czas pracy całego algorytmu. Ostatecznym efektem tego były zdarzające się błędne detekcje czerwonych okręgów wśród elementów otoczenia w sytuacji, gdy w polu widzenia brakowało właściwego celu.

Z perspektywy czasu za błąd można uznać jeden z aspektów sposobu pracy nad algorytmem wykrywania obrazu. Niemal przez cały okres rozwoju był on testowany wyłącznie na komputerze klasy PC, na podstawie pewnej stałej bazy zdjęć. Jednak finalne testy, wykonywane już na \textit{Raspberry Pi} uwydatniły dwa błędy takiego podejścia. Po pierwsze mimo starań o wszechstronność testowej bazy zdjęć nie dało się przewidzieć wszystkich możliwych przypadków, a o co za tym idzie algorytm przy niektórych z nich zachowywał się w sposób niespodziewany. Drugim problemem, który wyszedł na jaw dopiero w ostatniej fazie testów był brak wcześniejszego sprawdzenia powtarzalność pracy dla minimalnie odmiennych warunków. Każde zdjęcie w bazie testowej znacząco się różniło od innych, a z prawidłowej pracy algorytmu na jednym z nich wyciągano błędny wniosek na temat ogólnej skuteczności algorytmu w przedstawionych na zdjęciu warunkach.
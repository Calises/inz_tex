\namedchapter{Podsumowanie}
Celem pracy była budowa robota mobilnego typu czołg oraz implementacja algorytmu sterującego. Jego zadaniem było zlokalizowanie czerwonego koła na podstawie obrazu z kamery. Realizacja projektu wymagała:
\begin{itemize}
\item doboru elementów wykonawczych robota (silniki, serwomechanizmy) pod kątem założeń projektowych oraz wybór platformy, na której oparty zostanie system wizyjny,
\item zaprojektowania konstrukcji robota, wykonania modelu 3D oraz jego fizyczna realizacja,
\item stworzenia projektu sterownika silników napędowych, wykonania płytki drukowanej oraz implementacji regulatora,
\item zapewnienia komunikacji pomiędzy system wizyjnym a mikrokontrolerem,
\item zaimplementowania algorytmu sterującego robotem,
\item przygotowania programu służącego do analizy obrazu pozyskanego z kamery.
\end{itemize}

Każdy z powyższych etapów pracy kończony był testami pojedynczych modułów. Jednak dopiero po zrealizowaniu całości prac przystąpiono do testów, oceniających całością pracę robota. Testy polegały na pozostawianiu w różnych miejscach pomieszczenia arkuszy formatu A4 z czerwonymi kołami różnych rozmiarów i obserwacji zachowania robota, który miał ich poszukiwać. Parametry pracy podlegające ocenie to: szybkość działania, dokładność sterowania oraz skuteczność wykrywania okręgu.





%na czym polegał test, co podlegało ocenie
%sugestia czemu to działa tak a nie inaczej
%jakby to poprawic
%czy coś byś zmienił robiąc on nowa

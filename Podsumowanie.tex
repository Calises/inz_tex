\namedchapter{Podsumowanie}
Celem pracy była budowa robota mobilnego typu czołg oraz implementacja algorytmu sterującego. Jego zadaniem było zlokalizowanie czerwonego koła na podstawie obrazu z kamery. Realizacja projektu wymagała:
\begin{itemize}
\item doboru elementów wykonawczych robota (silniki, serwomechanizmy) pod kątem założeń projektowych oraz wybór platformy, na której oparty zostanie system wizyjny,
\item zaprojektowania konstrukcji robota, wykonania modelu 3D oraz jego fizycznej realizacji,
\item stworzenia projektu sterownika silników napędowych, wykonania płytki drukowanej oraz implementacji regulatora,
\item zapewnienia komunikacji pomiędzy system wizyjnym a mikrokontrolerem,
\item przygotowania programu służącego do analizy obrazu pozyskanego z kamery,
\item zaimplementowania algorytmu sterującego robotem.
\end{itemize}
Zakończenie każdego z etapów kończyło się weryfikacją, czy spełnia on założenia projektowe. Tak więc w przypadku konstrukcji mechanicznej należało przede wszystkim sprawdzić jakość dopasowania poszczególnych elementów. Po wykonaniu płytki drukowanej w pierwszej kolejności sprawdzone zostały połączenia pomiędzy komponentami (na wypadek gdyby któraś ze ścieżek była przerwana, bądź element źle zlutowany) oraz pomierzyć napięcia w poszczególnych miejscach. Następnie możliwe było przejście do implementacji sterownika silników, którego wykonanie oceniane było poprzez analizę odpowiedzi skokowej silnika odczytanej przy pomocy mikrokontrolera oraz \textit{Raspberry Pi}. Kolejnym etapem było zapewnienie stałej łączności pomiędzy systemem wizyjnym oraz sterownikiem. W przypadku, gdy określoną sekwencję bitów udało się odczytać na Atmedze, podpunkt uznano za zrealizowany. Prace nad analizą obrazu można było prowadzić równolegle do pozostały zadań. Wynika to z faktu, iż platformę \textit{Raspberry Pi} oraz moduł \textit{Raspberry Pi Camera HD} wystarczyło jedynie połączyć ze sobą oraz odpowiednio przygotować środowisko pracy. Program analizujący obraz testowany był na wielu różnorodnych zdjęciach. Gdy osiągnął zadowalający poziom - rozpoczęły się prace nad integracją wszystkich poszczególnych systemów. Wszystkie poprzednie prace wykonane były z dużą starannością więc połączenie ich ze sobą nie przyniosło większych problemów. Ostatnim etapem były testy polegające na pozostawieniu w różnych miejscach pomieszczenia arkuszy formatu A4, na których znajdowały się czerwone koła o różnych rozmiarach, oraz obserwacja zachowania robota. Parametrami podlegającymi ocenie to: szybkość działania, dokładność sterowania oraz skuteczność wykrywania okręgu.

% tutaj opisz efekt :)
% sugestia czemu to działa tak a nie inaczej
% jakby to poprawic
% czy coś byś zmienił robiąc on nowa
